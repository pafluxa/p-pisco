%% Copyright 2009 Jeffrey D. Hein
%
% This work may be distributed and/or modified under the
% conditions of the LaTeX Project Public License, either version 1.3
% of this license or (at your option) any later version.
% The latest version of this license is in
%   http://www.latex-project.org/lppl.txt
% and version 1.3 or later is part of all distributions of LaTeX
% version 2005/12/01 or later.
%
% This work has the LPPL maintenance status `maintained'.
% 
% The Current Maintainer of this work is Jeffrey D. Hein.
%
% This work consists of the files 3dplot.sty and 3dplot.tex

%Description
%-----------
%3dplot.tex - an example file demonstrating the use of the 3dplot.sty package.

%Created 2009-11-07 by Jeff Hein.  Last updated: 2009-11-09
%----------------------------------------------------------

%Update Notes
%------------

%2009-11-07: Created file along with 3dplot.sty package


\documentclass{article}
\usepackage{tikz}   %TikZ is required for this to work.  Make sure this exists before the next line

\usepackage{tikz-3dplot} %requires 3dplot.sty to be in same directory, or in your LaTeX installation

\usepackage[active,tightpage]{preview}  %generates a tightly fitting border around the work
\PreviewEnvironment{tikzpicture}
\setlength\PreviewBorder{2mm}

% command for co-polarized
\newcommand{\co}{\mathbin{\|}}

\begin{document}
	
	%Angle Definitions
	%-----------------
	
	%set the plot display orientation
	%synatax: \tdplotsetdisplay{\theta_d}{\phi_d}
	\tdplotsetmaincoords{60}{110}
	
	%define polar coordinates for some vector
	%TODO: look into using 3d spherical coordinate system
	\pgfmathsetmacro{\rvec}{.8}
	\pgfmathsetmacro{\thetavec}{30}
	\pgfmathsetmacro{\phivec}{60}
	\pgfmathsetmacro{\psivec}{-25}
	
	%start tikz picture, and use the tdplot_main_coords style to implement the display 
	%coordinate transformation provided by 3dplot
	\begin{tikzpicture}[scale=5,tdplot_main_coords]
	
	%set up some coordinates 
	%-----------------------
	\coordinate (O) at (0,0,0);
	
	%determine a coordinate (P) using (r,\theta,\phi) coordinates.  This command
	%also determines (Pxy), (Pxz), and (Pyz): the xy-, xz-, and yz-projections
	%of the point (P).
	%syntax: \tdplotsetcoord{Coordinate name without parentheses}{r}{\theta}{\phi}
	\tdplotsetcoord{P}{\rvec}{\thetavec}{\phivec}
	
	%draw figure contents
	%--------------------
	
	%draw the main coordinate system axes
	\draw[thick,color=blue,->] (0,0,0) -- (1,0,0) node[anchor=north east]{$\hat{\theta}'$};
	\draw[thick,color=blue,->] (0,0,0) -- (0,1,0) node[anchor=north west]{$\hat{\phi}'$};
	\draw[thick,color=blue,->] (0,0,0) -- (0,0,1) node[anchor=south]%
	{$\hat{p}'$};
	
	%draw a vector from origin to point (P) 
	\draw[thick,color=black,->] (O) -- (P) node[anchor=south]{$\hat{k}$};
	
	%draw projection on xy plane, and a connecting line
	\draw[dashed, color=red] (O) -- (Pxy);
	\draw[dashed, color=red] (P) -- (Pxy);
	
	%draw the angle \phi, and label it
	%syntax: \tdplotdrawarc[coordinate frame, draw options]{center point}{r}{angle}{label options}{label}
	\tdplotdrawarc{(O)}{0.3}{0}{\phivec}{anchor=north}{$\sigma$}
	
	
	%set the rotated coordinate system so the x'-y' plane lies within the
	%"theta plane" of the main coordinate system
	%syntax: \tdplotsetthetaplanecoords{\phi}
	\tdplotsetthetaplanecoords{\phivec}
	
	%draw theta arc and label, using rotated coordinate system
	\tdplotdrawarc[tdplot_rotated_coords]{(0,0,0)}{0.5}{0}{\thetavec}{anchor=south west}{$\rho$}
	
	%draw some dashed arcs, demonstrating direct arc drawing
	\draw[dashed,tdplot_rotated_coords] (\rvec,0,0) arc (0:90:\rvec);
	\draw[dashed] (\rvec,0,0) arc (0:90:\rvec);
	
	%set the rotated coordinate definition within display using a translation
	%coordinate and Euler angles in the "z(\alpha)y(\beta)z(\gamma)" euler rotation convention
	%syntax: \tdplotsetrotatedcoords{\alpha}{\beta}{\gamma}
	\tdplotsetrotatedcoords{\phivec}{\thetavec}{0}
	
	%translate the rotated coordinate system
	%syntax: \tdplotsetrotatedcoordsorigin{point}
	\tdplotsetrotatedcoordsorigin{(P)}
	
	%use the tdplot_rotated_coords style to work in the rotated, translated coordinate frame
	% draw hat_p_0
	%\draw[thick,tdplot_rotated_coords,color=blue,->] (0,0,0) -- (0,0,.4) node[anchor=south]{$\hat{p}'$};
	% draw hat_theta
	\draw[thick,tdplot_rotated_coords,->] (0,0,0) -- (.3,0,0) node[anchor=south west]{$\hat{\rho}$};
	% draw hat_phi
	\draw[thick,tdplot_rotated_coords,->] (0,0,0) -- (0,.3,0) node[anchor=west]%
	{$\hat{\sigma}$};
	
	%rotate coordinate system by psi
	\tdplotsetrotatedcoords{\phivec}{\thetavec}{45}
	
	% draw hat_e_co
	%\draw[thick,tdplot_rotated_coords,color=blue,->] (0,0,0) -- (.3,0,0) node[anchor=south]{$\hat{e}_{\co}$};
	% draw hat_phi_prime
	%\draw[thick,tdplot_rotated_coords,color=blue,->] (0,0,0) -- (0,0.3,0) node[anchor=west]{$\hat{\phi}'$};
	
	% draw arc to show angle psi
	%\tdplotsetrotatedcoordsorigin{(P)}
	%\tdplotsetrotatedcoords{\phivec}{\thetavec}{0}
	%\tdplotdrawarc[tdplot_rotated_coords]{(0,0,0)}{0.3}{0}{\psivec}{anchor=north}{$\psi$}
	
	%WARNING:  coordinates defined by the \coordinate command (eg. (O), (P), etc.)
	%cannot be used in rotated coordinate frames.  Use only literal coordinates.  
	
	%draw some vector, and its projection, in the rotated coordinate frame
	%\draw[-stealth,color=blue,tdplot_rotated_coords] (0,0,0) -- (.2,.2,.2);
	%\draw[dashed,color=blue,tdplot_rotated_coords] (0,0,0) -- (.2,.2,0);
	%\draw[dashed,color=blue,tdplot_rotated_coords] (.2,.2,0) -- (.2,.2,.2);
	
	%show its phi arc and label
	%\tdplotdrawarc[tdplot_rotated_coords,color=blue]{(0,0,0)}{0.2}{0}{45}{anchor=north west,color=black}{$\phi'$}
	
	%change the rotated coordinate frame so that it lies in its theta plane.
	%Note that this overwrites the original rotated coordinate frame
	%syntax: \tdplotsetrotatedthetaplanecoords{\phi'}
	%\tdplotsetrotatedthetaplanecoords{45}
	
	%draw theta arc and label
	%\tdplotdrawarc[tdplot_rotated_coords,color=blue]{(0,0,0)}{0.2}{0}{55}{anchor=south west,color=black}{$\theta'$}
	
	\end{tikzpicture}
	
\end{document}